\section{Project Results}

I mentioned before that it has been only three months that I have been working at \gls{nist} as a member of the \gls{samate} team. It is therefore too early to make a relevant reflection upon my work.

So far, I learned a lot about static analysis and vulnerability addition techniques. I read a lot of publications and parts of books so I could draw an enlightened state of the art. This allowed me not only to understand what motivates my project, and realize the various opportunities or issues behind it, but also it built for me a solid knowledge foundation on which I could build upon thoughts and ideas that led to the final design of my solution proposal.

At this time, I have been able to setup a working \gls{lava} environment, which proved itself to be quite tricky to handle. I had the time to run some tests in this environment, so far, mostly on the provided examples like the \emph{file} program. Once I got a complete and working \gls{lava} system, I finally got the opportunity to study the injected bugs, and find where the information about the \acrfull{dua} and \acrfull{atp} was stored. I could therefore design a SQL request to extract the relevant information from the \gls{lava} POSTGRES database. At this point, we ensured that \gls{lava} can be used to point out the \glspl{dua} and \glspl{atp} locations in the source code for later manual bug injection. This is a first very promising result for the future of my project.