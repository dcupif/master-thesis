Project management in research environment is just as important as in companies. Without any methodology or process, you cannot achieve any consistent work. A research project needs to be organized and managed just like any other project.

Nonetheless, research projects are not like any other kind of projects, so you should not use the same processes that are used for software development projects in companies (e.g., \emph{Agile} methods such as \emph{Scrum} to cite the most popular.). Research projects need to be managed in a very flexible way, because you should be able to nurture your creativity and let new ideas come into the game as you go.

But flexibility is not synonymous with disorder. It actually takes a lot of discipline to manage a research project. The most challenging and frequently encountered issue in research laboratories, is to keep track of your activity: what did you experiment that day? What was your last great idea about the project? How does this tool work, I used it two months ago but now I cannot remember anything about it? It is amusing how computers provide a lot of tools to keep track of your activity, and still computer science researchers seem to be having a lot of problems around this issue. When you work on a research project, you must log everything you do, document every tool you use, or that you develop. It is incredible to realize how great ideas can be forgotten if they are not well-documented. Considering the above, I dedicated some of my everyday time to log my activity, and keep important thoughts on clear and accessible supports. This helped me a lot to edit this master thesis.

I loved working at \gls{nist} as a member of the \gls{samate} team. I was really independent and self-governing in my work as I was the only one to work on this project for the first three months. This allowed me to choose the way I wanted to work, and to submit my ideas all along the internship. We did not have any kind of specific project management method to follow my progress, but I used to talk a lot with my co-workers about my project to get their feedback almost everyday. Once a week, we had a team meeting to speak about our own projects, or even present some new technologies one of us could have found during the week. Moreover, the team has a wiki available to share some important information between co-workers.

Finally, for the above reasons, it was quite difficult to schedule an action plan because when I first arrived in the team, there was just no such thing. I had to build something up from scratch. It is only when I had made enough research about the project, that I could come up with an action plan and try to stick to it. It proved to be working pretty well so far.