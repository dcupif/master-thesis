\vspace{-0.5cm}

\section{Project Future}

According to my action plan schedule, the next step is to make \gls{lava} run on future potential test cases. We already know from the \gls{lava} team that it is possible (because designed for it), but we might need to configure the base programs to be compiled with the CLANG compiler embedded in \gls{lava}. We still need to investigate more to know how complex it would be to run \gls{lava} on other pieces of software. But, according to the \gls{lava} development team, it should be quite easy to do so.

There is still so much more to achieve on this project. Even if we prove that our proposal method for injecting bugs saves enough time to build high quality vulnerability corpora, keep in mind that \gls{lava} method to find the attack points is for the moment focused on locations in the code prone to buffer overflows. If we want to comply with our initial requirements, we will need to figure out ways to seed different types of vulnerabilities. Moreover, the latest implementation \gls{lava} only supports the C language. We want to produce at least C and Java test cases, so this will be an issue to deal with later.