Through this document, we have demonstrated that static analyzers have the ability to enhance software security and software assurance all along the development lifecycle. Assessing the performance of these tools is crucial to measure to which extent they can guide developer's work and save time for code reviews.

My project, which was related to the development of test material for assessing static analysis tools, has an indirect economic impact on the software industry. Indeed, assessing static analyzers means helping to develop better tools and therefore contributes to nourish \emph{hill climbing}. Static analyzers can actually save a lot of money and time for companies as they prove themselves to be the most efficient at the beggining of the development lifecycle. Finding an error early in the development lifecycle reduces the cost of fixing it.

At a high level picture, finding and fixing bugs is an insanely time consuming activity. Some studies report that debugging takes between 40\% to 75\% developer's time \cite{dolan2016million}. And not only does debugging cost a lot of time and money to companies, but the consequences on the users can be dreadful. \gls{nist} has expended its contribution in helping the community find better and faster ways to automatically identify and fix bugs. Considering that \gls{nist} missions are closely bound to the industrial sector, this project is definitely destined to help companies build secure software.

The technological impact of my solution will not be a state of the art breakthrough in vulnerability addition techniques. We cannot wait for the IT community to make a great discovery in automated vulnerability addition techniques. The need for software assurance is critical. However, if we can achieve producing such a high quality vulnerability corpora following my proposal technique, which is a combination of manual and automated methods, the resulting corpora would be a significant resource for assessing static analyzers. As stated previously, we cannot expect to develop better static analysis tools if we are not capable of assessing their effiency. Developing tools without knowing how good they are, is like groping our way forward blindly.