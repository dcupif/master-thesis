\section{Aim of the Project}

Although the test material for the previous \gls{sate}s is of high quality, combining three types of test cases is inconvenient. The review of all of the tools' reports requires a lot of resources, because each type of test case undergoes a different review process. In addition, the conclusions are incomplete, because they are drawn from an interpretation of results obtained from the aggregation of imperfect test cases.

In preparation for its sixth edition, \gls{sate} needs improved test material: a large code base of realistic software with clearly identified \glspl{weakness}. We already know that producing such test data entails a colossal amount of work, but whatever time it would take, the ability to produce such \gls{vulnerability} corpora would be ideal for the IT community. Test material is essential for analyzing the capabilities of static analyzers. Such analyses would enable the development of better tools by providing a more accurate insight on tools' performance and behavior.

My work as a research associate on the \gls{samate} team was to develop this new revised and upgraded test material. The goal of this project was to figure out how to produce tests cases exhibiting all three characteristics ground truth, relevance, and statistical significance.

Manually producing such a \gls{vulnerability} corpora is infeasible. This difficult, lengthy process will be discussed in more detail in Section \ref{sec:introduction-of-vulnerabilities-to-code}. During my internship, I had the opportunity to study and research new techniques for automating parts of the process of producing high quality \gls{vulnerability} corpora.